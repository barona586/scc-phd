The School's PhD appraisal schedule is designed to assist in the monitoring of progress towards timely submission, and providing feedback and assessment above and beyond that provided through usual supervision.

The appraisal schedule briefly comprises of a series \hyperref[sec:points]{appraisal points}, all of which require a minimum of an updated \hyperref[sec:report]{progress report} to be submitted beforehand. Some \hyperref[sec:points]{appraisal points} require further documents to be submitted, and some require a meeting with the \hyperref[sec:panel]{appraisal panel}. The \hyperref[sec:panel]{appraisal panel} will provide \hyperref[sec:feedback]{feedback} at each \hyperref[sec:points]{appraisal point}.

This schedule is in line with the Faculty of Science and Technology PhD assessment, and adheres to
\href{https://gap.lancs.ac.uk/ASQ/QAE/MARP/Documents/PGR-Regs.pdf}{Lancaster University's Postgraduate Research Regulations} and 
\href{https://gap.lancs.ac.uk/ASQ/Policies/Documents/Postgraduate-Research-Code-of-Practice.pdf}{The Code of Practice for Postgraduate Research Programmes}.

\section{Appraisal Panel} \label{sec:panel}
Each PhD student will be assigned an appraisal panel at the start of the PhD. The panel will consist of:
\begin{description}
	\item[A Subject specialist(s)] A member of academic staff who is relatively close to the PhD research, but not involved in direct supervision. This could be a named supervisor, somebody who might be involved in collaborative work, or somebody from the same research group. In the event of an inter-disciplinary PhD (e.g.\ with supervisors from multiple departments) it is expected that there will be one panel member from each discipline.
	\item[A PGRC member] A member of academic staff from the SCC Postgraduate Research Committee, whose responsibility will be to ensure consistency and fairness across appraisals, and feedback issues and concerns to the School. This panel member shall normally be from a different research group to that of the PhD student and subject specialist.
\end{description}

The PhD student should feel welcome to discuss questions and problems with panel members outside of the appraisals. During panel meetings there shall be an explicit discussion about supervision and feedback on School processes. All such discussions are held under the \href{https://www.chathamhouse.org/chatham-house-rule}{Chatham House Rule}.

Where possible, the appraisal panel will remain the same throughout the PhD, but replacements will be made if required (e.g.\ due to staff departures).

Please note, appraisal panel members are not normally permitted to serve as an examiner for the final PhD submission and viva.

\section{Progress Report} \label{sec:report}
The progress report, to be established in the first 4 months of the PhD, should be a single evolving document for the whole PhD process, and updated regularly. As a minimum, for every \hyperref[sec:points]{appraisal point} that follows, an updated copy of the report should be uploaded to the \href{https://modules.lancaster.ac.uk/course/view.php?id=7050}{Moodle space}, containing the following:
\begin{itemize}
	\item A brief summary (around half a page) of progress since the last appraisal stage and against the completion timeline, and any problems encountered.
	\item Updated assessment of training and development of skills needed, and a record of any training or development undertaken, including courses, presentations, teaching, etc.\
	\item An updated completion timeline with appropriate milestones and publication plans for the remainder of the PhD.
\end{itemize}

A \LaTeX{} template is available for the progress report. In addition, extra text or documents will be requested at different appraisal stages. PhD students are also welcome to submit alongside any papers submitted in the appraisal period.


\section{Feedback} \label{sec:feedback}
At each appraisal stage, the \hyperref[sec:panel]{panel} shall provide feedback on the PhD research, progress, and planning. The subject specialist shall lead feedback on the research element, with the PGRC member leading feedback on the progress and planning. It is expected that supervisors will provide feedback through the normal supervision process. Supervisors are also expected to provide feedback on appraisal documents before submission, for which the PhD student and supervisors should agree a timeline in advance.

Feedback from each appraisal stage shall be provided within 4 weeks after the appraisal due date, document submission, or a related panel meeting (whichever is the latest). All written feedback will be provided via the \href{https://modules.lancaster.ac.uk/course/view.php?id=7050}{Moodle space}.